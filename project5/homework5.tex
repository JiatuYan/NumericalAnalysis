\documentclass[twoside,a4paper]{article}
\usepackage{geometry}
\geometry{margin=1.5cm, vmargin={0pt,1cm}}
\setlength{\topmargin}{-1cm}
\setlength{\paperheight}{29.7cm}
\setlength{\textheight}{25.3cm}

% useful packages.
\usepackage{amsfonts}
\usepackage{amsmath}
\usepackage{amssymb}
\usepackage{amsthm}
\usepackage{enumerate}
\usepackage{graphicx}
\usepackage{multicol}
\usepackage{fancyhdr}
\usepackage{layout}

% some common command
\newcommand{\dif}{\mathrm{d}}
\newcommand{\avg}[1]{\left\langle #1 \right\rangle}
\newcommand{\difFrac}[2]{\frac{\dif #1}{\dif #2}}
\newcommand{\pdfFrac}[2]{\frac{\partial #1}{\partial #2}}
\newcommand{\OFL}{\mathrm{OFL}}
\newcommand{\UFL}{\mathrm{UFL}}
\newcommand{\fl}{\mathrm{fl}}
\newcommand{\op}{\odot}
\newcommand{\Eabs}{E_{\mathrm{abs}}}
\newcommand{\Erel}{E_{\mathrm{rel}}}

\begin{document}

\pagestyle{fancy}
\fancyhead{}
\lhead{NAME Jiatu Yan}
\chead{Numerical Analysis homework \#5}
\rhead{Date}


\section*{I. \small{Deduce Corollary 4.24 from the theorem of derivatives of B-splines.}}

Since $\forall i, \frac{d}{dx}B_{i}^{n+1}\left( x \right) $ has primitive on $\mathbb{R}\backslash \{t_{n}\}$
, $\{t_{n}\}$ are isolated with each other, so we have 
\[\int_{t_{i-1}}^{t_{i+n+1}}\frac{d}{dx}B_{i}^{n+1}\left( x \right)dx
	=B_{i}^{n+1}\left( t_{i+n+1} \right)-B_{i}^{n+1}\left(t_{i-1}  \right)=0-0=0  
.\] 
So by the theorem of derivatives of B-splines, and the support of $B_{i}^{n}\left( x \right)=[t_{i-1},t_{i+n}] $, we have
\begin{equation*}
	\begin{split}
	\int_{t_{i-1}}^{t_{i+n+1}}\frac{d}{dx}B_{i}^{n+1}\left( x \right)dx 
	&=\int_{t_{i-1}}^{t_{i+n+1}}\frac{\left( n+1 \right) B_{i}^{n}\left( x \right) }{t_{i+n}-t_{i-1}}dx
	-\int_{t_{i-1}}^{t_{i+n+1}}\frac{\left( n+1 \right)B_{i+1}^{n}\left( x \right)  }{t_{i+n+1}-t_{i}}dx=0\\
	&\Longrightarrow \frac{n+1}{t_{i+n}-t_{i-1}}\int_{t_{i-1}}^{t_{i+n}}B_{i+1}^{n}\left( x \right)dx
	=\frac{n+1}{t_{i+n+1}-t_{i}}\int_{t_{i-1}}^{t_{i+n+1}}B_{i+1}^{n}\left( x \right)dx =C\\.
        \end{split}
\end{equation*}

Because $\{t_{n}\}$ are randomly taken, by changing $t_{i}$ and $t_{i-1}$ separately, we can conclude that C is independent with $\{t_{n}\}$.
So we have 
\[ 
	\forall i\quad, \frac{1}{t_{i+n} - t_{i-1}}\int_{t_{i-1}}^{t_{i+n}}B_{i}^{n}\left( x \right)dx =\frac{C}{n+1}
.\]

The intergral is independent with its index.

\section*{II. \small{Symmetric Polynomials.}}

\subsection*{II-a. \small{Verify the theorem 4.34 for $m=4$ and  $n=2$.}}

We cam make a tabular for the divided difference of $x^{m}$. 

\begin{tabular}{c|cccc}
x &\\
$x_{1}$&$x_{1}^{4}$\\
$x_{2}$&$x_{2}^{4}$&$x_{2}^{3}+x_{2}^{2}x_{1}+x_{2}x_{1}^{2}+x_{1}^{3}$\\
$x_{3}$&$x_{3}^{4}$&$x_{3}^{3}+x_{3}^{2}x_{2}+x_{3}x_{2}^{2}+x_{2}^{3}$&
$ \frac{x_{3}^{3}-x_{1}^{3}+x_{2}\left(x_{3}^2-x_{1}^2\right)+x_{2}^2\left(x_{3}-x_{1}\right)}{x_{3}-x_{1}}
=x_{1}^{2}+x_{2}^2+x_{3}^2+x_{1}x_{2}+x_{1}x_{3}+x_{2}x_{3}$\\
\end{tabular}

By the definition of complete symmetric polynomial, we have
 \[
	 \tau_{4-2}\left( x_{1},x_{2},x_{3} \right)=x_{1}^{2}+x_{2}^{2}+x_{3}^2+x_{1}x_{2}+x_{1}x_{3}+x_{2}x_{3}
	 =[x_{1},x_{2},x_{3}]x^{4}
.\] 

\subsection*{II-b. \small{Prove this theorem by the lemma on the recursive relation on complete symmetric polynomials.}}

Don't lose generality, we can let $i=1$.

Firstly, we prove  $\left( x_{n+1}-x_{1} \right)\tau_{k}\left( x_{1},\ldots,x_{n+1} \right)
=\tau_{k+1}\left( x_{2},\ldots,x_{n+1} \right)-\tau_{k+1}\left( x_{1},\ldots,x_{n} \right)  $ 
\begin{equation*}
	\begin{split}
		LHS&=x_{n+1}\tau_{k}\left( x_{1},\ldots,x_{n+1} \right)-x_1\tau_{k}\left( x_1,\ldots,x_{n+1} \right)\\
		   &=\tau_{k+1}\left( x_1,\ldots,x_{n+1} \right)-\tau_{k+1}\left( x_{1},\ldots,x_{n} \right)
		   -[\tau_{k+1}\left( x_{1},\ldots,x_{n+1} \right)-\tau_{k+1}\left( x_{2},\ldots,x_{n+1} \right)] \\
		   &=\tau_{k+1}\left( x_2,\ldots,x_{n+1} \right)-\tau_{k+1}\left( x_{1},\ldots,x_{n} \right)\\
		   &=RHS.
	\end{split}
\end{equation*}

When $n=0$, by the definition, we have
 \[
	 \forall m\quad, \tau_{m}\left( x_{1} \right)=x_{1}^{m}=[x_{1}]x^{m} 
.\] 
So when $n=0$, the theorem is right.

We randomly take a m, we suppose when $n=k<m$, the equation 
$\tau_{m-k}\left( x_1,\ldots,x_{1+k} \right)=[x_1,\ldots,x_{1+k}]x^{m} $ exists.
Then when $n=k+1$, we have

\begin{equation*}
	\begin{split}
		[x_1,\ldots,x_{k+2}]x^{m}&=\frac{[x_{2},\ldots,x_{2+n}]x^{m}-[x_1,\ldots,x_{n+1}]x^{m}}{x_{1+n}-x_{1}}\\
		&=\frac{1}{x_{1+n}-x_{1}}\left( \tau_{m-n}\left( x_2,\ldots,x_{n+2} \right)
		-\tau_{m-n}\left( x_1,\ldots,x_{n+1} \right)   \right)\\
		&=\tau_{m-\left( n+1 \right) }\left( x_1,\ldots,x_{n+2} \right). 
	\end{split}	
\end{equation*}

So $\forall m\in \mathbb{N}^{+},i\in\mathbb{N},\forall n=0,1,\ldots,m,
\tau_{m-n}\left( x_i,\ldots,x_{i+n} \right)=[x_i,\ldots,x_{i+n}]x^{m}$.
\end{document}

%%% Local Variables: 
%%% mode: latex
%%% TeX-master: t
%%% End: 
