\documentclass[twoside,a4paper]{article}
\usepackage{geometry}
\geometry{margin=1.5cm, vmargin={0pt,1cm}}
\setlength{\topmargin}{-1cm}
\setlength{\paperheight}{29.7cm}
\setlength{\textheight}{25.3cm}

% useful packages.
\usepackage{amsfonts}
\usepackage{amsmath}
\usepackage{amssymb}
\usepackage{amsthm}
\usepackage{enumerate}
\usepackage{graphicx}
\usepackage{multicol}
\usepackage{fancyhdr}
\usepackage{layout}

% some common command
\newcommand{\dif}{\mathrm{d}}
\newcommand{\avg}[1]{\left\langle #1 \right\rangle}
\newcommand{\difFrac}[2]{\frac{\dif #1}{\dif #2}}
\newcommand{\pdfFrac}[2]{\frac{\partial #1}{\partial #2}}
\newcommand{\OFL}{\mathrm{OFL}}
\newcommand{\UFL}{\mathrm{UFL}}
\newcommand{\fl}{\mathrm{fl}}
\newcommand{\op}{\odot}
\newcommand{\Eabs}{E_{\mathrm{abs}}}
\newcommand{\Erel}{E_{\mathrm{rel}}}

\begin{document}

\pagestyle{fancy}
\fancyhead{}
\lhead{NAME JiaTu Yan}
\chead{Numerical Analysis homework \#0}
\rhead{Date 2020.3.7}


\section*{I. Exercise 0.10 \small{each of the following statements into logical statements}}

\subsection*{I-a \small{The only even prime is 2}}

Let $\mathbb{P} \subseteq \mathbb{N}^+$ denote the set of prime numbers. 
$\exists! x = 2 \in \mathbb{P}, s.t. \frac{x}{2} \in \mathbb{Z}.$ 

\subsection*{I-b \small{Multiplication of integers is associative.}}
$\forall x,y,z \in \mathbb{Z}, x(yz)=(xy)z$

\subsection*{I-c \small{Goldbath's conjecture has at most a finite numberof counterexamples.}}

Let $\mathbb{P}$ denote the set of counterexamples of Goldbath's conjecture.
$\forall$ a sequence $\{ x_{n} \}, x_{n} \in \mathbb{P}, n \in \mathbb{N}^+$.
$\exists \mathbb{N} s.t. \forall n \geq \mathbb{N}, 
\exists m < \mathbb{N} s.t. x_{n} = x_{m}$.  

\section*{II. \small{Prove: On $(a, \infty)$, $f(x)=\frac{1}{x^2}$ is uniformly continuous if $a>0$ and is not so if $a=0$  } }
Proof:

$\forall \epsilon > 0, \exists \delta = \frac{a^{3}\epsilon}{2},
s.t. \forall x, y \in (a, \infty), \mid x - y \mid < \delta$
we can suppose that $x > y$.
Then, we have $\mid\frac{1}{x^2} - \frac{1}{y^2}\mid \leq \mid\frac{2x\delta}
{y^4}\mid \leq \frac{2\delta}{a^3} < \epsilon$.
Thus, $\frac{1}{x^2}$ is uniformly continuous.

\subsection*{\small{$\textcircled{2}$ $a=0$}}

$\forall \delta > 0$, $\exists x=\frac{\delta}{2} y=\delta$, $\exists \epsilon = \frac{2}{\delta^2}$.
Thus, $\mid x - y \mid < \delta$, $\mid \frac{1}{x^2} - \frac{1}{y^2} \mid = \frac{3}{\delta^2} > \epsilon$
So, when $a = 0$, $\frac{1}{x^2}$ is not uniformly continous.

\section*{III. \small{Prove: $d(x,y) = \sum_{j=1}^{n}\frac{1}{2^j}\frac{\mid\xi_{j} - \eta_{j}\mid}{1 + \mid\xi_{j} - \eta_{j}\mid}$ is a metrix on $\mathcal{X}$}}
Proof:

\subsection*{\small{$\textcircled{1}$ non-negativity}}

Since $\mid\xi_{j} - \eta_{j}\mid \geq 0$, $\forall x,y \in \mathcal{X}$, $d(x,y) \geq 0$.

\subsection*{\small{$\textcircled{2}$ identity of indiscernibles}} 

Since $\mid\xi_{j} - \eta_{j}\mid \geq 0$, and iff $\xi_{j} = \eta_{j}$, the equality is reached. Each item of the sequence $d(x,y)$ is bigger than 0.
We have $d(x,y) = 0 \iff \forall j \in \mathbb{N}^+, \xi_{j} = \eta_{j}$, thus $d(x,y) \iff x = y$. 

\subsection*{\small{$\textcircled{3}$ symmetry}}

Since $\mid\xi_{j} - \eta_{j}\mid = \mid\eta_{j} - \xi_{j}\mid$, $d(x,y) = d(y,x)$ is obvious.

\subsection*{\small{$\textcircled{4}$ triangle inequality}}

$\forall x,y,z \in \mathbb{X}$, $x = (\xi_{j})$, $y = (\eta_{j})$, $z = (\delta_{j})$.
Suppose $a_{j} = \mid\xi_{j} - \eta_{j}\mid$, $b_{j} = \mid\xi_{j}- \delta_{j}\mid$, $c = \mid\eta_{j} - \delta_{j}\mid$.
Since $\forall j \in \mathbb{N}^+ a_{j}, b_{j}, c_{j} > 0$. $\forall j$ and fix it. In the following proof, I will omit j.
Because $a \leq b + c$, we have $a + ba + ca + abc \leq b + c + 2bc + ab + ac + 2abc$, sort it we will have $(1 + a)[b(1 + c) + c(1 + b)] \geq a(1 + b)(1 + c)$.
Since $a + 1 \geq 0$, we will have $0 = \frac{a}{1 + a} \leq \frac{b}{1 + b} + \frac{c}{1 + c}$. Obviously, we can have $d(x,y) \leq d(x,z) + d(y,z)$.

\section*{IV. \small{Prove: $d(x,y) = (\sum_{i=1}^{\infty}{\mid\xi_{i} - \eta_{i}\mid}^{p})^{\frac{1}{p}}$ is a metrix}}

Proof:

The non-negativity, identity of indiscernibles and symmetry is obviously proved by $\mid\xi_{i} - \eta{i}\mid \geq 0$.
Now prove trangle inequality.
Firstly, we prove the Holder inequality.
p,q are conjugate exponents.Since we have $\forall \alpha,\beta \in \mathbb{R}^+$, $\alpha\beta \leq \frac{\alpha^p}{p} + \frac{\beta^q}{q}$.
Suppose $A_{i} = \frac{\mid\xi_{i}\mid}{(\sum_{i=1}^{\infty}{{\mid\xi_{i}\mid}^p})^{\frac{1}{p}}}$, $B_{i} = \frac{\mid\eta_{i}\mid}{(\sum_{i=1}^{\infty}{{\mid\eta_{i}\mid}^q)}^{\frac{1}{q}}}$. 
So we have $A_{i}B_{i} \leq \frac{{A_{i}}^p}{p} + \frac{{B_{i}}^q}{q}$. Now sum up the inequality by i, We have 
$$\frac{\sum_{i=1}^{\infty}\mid\xi_{i}\eta_{i}\mid}{({(\sum_{i=1}^{\infty}{{\mid\xi_{i}\mid}^{p}})}^{\frac{1}{p}})
({({\sum_{i=1}^{\infty}{\mid\eta_{i}\mid}^q})}^{\frac{1}{q}})} \leq \frac{1}{p} + \frac{1}{q} = 1$$.
Then we use Holder inequality to prove Minkowski inequality.
$$A = (\sum_{i=1}^{\infty}{\mid\xi_{i} + \eta_{i}\mid}^{p})^{\frac{1}{p}} 
\leq (\sum_{i=1}^{\infty}\mid \xi_{i}+\eta_{i}\mid\mid\xi_{i}+\eta_{i}\mid^{p-1})^{\frac{1}{p}}$$
By Holder inequality we have
$$A \leq (\sum_{i=1}^{\infty}\mid\xi_{i}+\eta_{i}\mid^{p})^{\frac{1}{p}}
(\sum_{i=1}^{\infty}\mid\xi_{i}+\eta_{i}\mid^{pq-q})^{\frac{1}{q}}
\leq(\sum_{i=1}^{\infty}{\mid\xi_{i}\mid}^{p})^{\frac{1}{p}}[\sum_{i=1}^{\infty}{\mid\xi_{i}+\eta_{i}\mid}^{pq-q}]^{\frac{1}{q}} +
(\sum_{i=1}^{\infty}{\mid\eta_{i}\mid}^{p})^{\frac{1}{p}}
[\sum_{i=1}^{\infty}{\mid\xi_{i}+\eta_{i}\mid}^{pq-q}]^{\frac{1}{q}}$$
Sort the inequality we have 
$$(\sum_{i=1}^{\infty}{\mid\xi_{i} + \eta_{i}\mid}^{p})^{\frac{1}{p}} = (\sum_{i=1}^{\infty}{\mid\xi_{i} + 
\eta_{i}\mid}^{p})^{\frac{1}{p}(p - \frac{p}{q})} \leq (\sum_{i=1}^{\infty}{\mid\xi_{i}\mid}^{p})^{\frac{1}{p}}+
(\sum_{i=1}^{\infty}{\mid\eta_{i}\mid}^{p})^{\frac{1}{p}}$$
Then $\forall x,y,z \in \mathbb{X}, x=\{x_{i}\}, y=\{y_{i}\}, z=\{z_{i}\}$.
Replace $\xi{i}$ by $x_{i} - z_{i}$, $\eta_{i}$ by $y_{i} - z{i}$, thus $d(x,y) \leq d(x,z) + d(y,z)$.
So that we proved that it is a metrix.

\section*{V. \small{Deduce additivity and conjugate homogeneity in the second slot}}

\subsection*{\small{$\textcircled{1}$ Aditivity}}

$$\forall u,v,w \in \mathcal{V}, \langle u,v+w\rangle=
\overline{\langle v+w,u\rangle}=
\overline{\langle v,u\rangle}+\overline{\langle w,u\rangle}=
\langle u,v\rangle+\langle u,w\rangle $$

\subsection*{\small{$\textcircled{2}$ Aditivity}}

$$\forall u,v\in \mathcal{V},\forall a \in \mathbb{F}, 
\langle u,av\rangle=
\overline{\langle av,u\rangle}=
\overline{a}\overline{\langle v,u\rangle}=
\overline{a}\langle u,v\rangle$$


\section*{VI. \small{In the case of Euclidean $\ell_{p}$norms
, show that the parallelogram law (0.72) holds if and only if $p=2$.}}
In the whole question, we discuss in the field $\mathbb{F}^{n}$, 
$1\leq n<\infty$

\subsection*{\small{$\textcircled{1}$$n=1$}}

$\forall u,v\in \mathbb{F},
\forall p\geq1,
2{\mid\mid u\mid\mid}^2+
2{\mid\mid v\mid\mid}^{2}=
2{({\mid u\mid}^p)}^{\frac{2}{p}}+
2{({\mid v\mid}^p)}^{\frac{2}{p}}=
2u^2+2v^2=(u-v)^2+(u+v)^2=
{({\mid u-v\mid}^p)}^{\frac{2}{p}}+
{({\mid u+v\mid}^p)}^{\frac{2}{p}}+
{\mid\mid u-v\mid\mid}^2+
{\mid\mid v+u\mid\mid}^{2}$

\subsection*{\small{$\textcircled{2}$$n>1,p\neq2$}}
We take $u=(u_{i}),v=(v_{i}),u_{1}=1,u_{2}=-1, v_{1}=1,v_{2}=1$,
if $i\neq1,2, u_{i}=v_{i}=0$(if $u_{i},v_{i}$ exists.)
Then, if $p=\infty$,
$$2{\mid\mid u\mid\mid}^{2}+2{\mid\mid v\mid\mid}^2=
2+2=4,{\mid\mid u+v\mid\mid}^2+{\mid\mid u+v\mid\mid}^2=
2^2+2^2=8\neq{\mid\mid u\mid\mid}^{2}+{\mid\mid v\mid\mid}^{2}$$
if $p<\infty$,
$$2{\mid\mid u\mid\mid}^{2}+2{\mid\mid v\mid\mid}^2=
2(2)^{\frac{2}{p}}+2(2)^{\frac{2}{p}}=
4(2)^{\frac{2}{p}},
{\mid\mid u+v\mid\mid}^{2}+{\mid\mid u-v\mid\mid}^{2}=
(2)^2+(2)^2=8\neq{\mid\mid u\mid\mid}^{2}+{\mid\mid v\mid\mid}^{2}$$
\subsection*{\small{$\textcircled{3}$$n>1,p=2$}}
We take $u=\{u_{i}\},v=\{v_{i}\}$
\begin{equation*}
	\begin{split}
2{\mid\mid u\mid\mid}^{2}+2{\mid\mid v\mid\mid}^2=
2{(\sum_{i=1}^{n}{(\mid u_{i}\mid)}^{2})}^{\frac{2}{2}}
+2{(\sum_{i=1}^{n}{(\mid u_{i}\mid)}^{2})}^{\frac{2}{2}}=
\sum_{i=1}^{n}(u_{i}^{2}+v_{i}^{2}+2u_{i}v_{i})
+\sum_{i=1}^{n}(u_{i}^{2}+v_{i}^{2}-2u_{i}v_{i})\\=
(\sum_{i=1}^{n}(u_{i}+v_{i})^{2})^{\frac{2}{2}}
+(\sum_{i=1}^{n}(u_{i}-v_{i})^{2})^{\frac{2}{2}}=
{\mid\mid u+v\mid\mid}^{2}+{\mid\mid u-v\mid\mid}^{2}
\end{split}
\end{equation*}
\section*{VII. \small{Prove Theorem 0.102}}
\subsection*{\small{$\textcircled{1}$ induced norm $\mid\mid.\mid\mid$ holds for some inner product$\langle,\rangle$}}
	
$\forall u,v\in \mathbb{F}^n$.We have
$$\mid\mid u-v\mid\mid^{2}=\langle u-v,u-v\rangle=
\langle u,u\rangle+\langle v,v\rangle
-\langle v,u\rangle-\langle u,v\rangle$$
$$\mid\mid u+v\mid\mid^{2}=\langle u+v,u+v\rangle=
\langle u,u\rangle+\langle v,v\rangle
+\langle v,u\rangle+\langle u,v\rangle$$
sum the two equation up, we have
$$2\mid\mid u\mid\mid^{2}+2\mid\mid v\mid\mid^{2}=
\mid\mid u-v\mid\mid^2+\mid\mid v+u\mid\mid^2$$

\subsection*{\small{$\textcircled{2}$ The parallelogram law holds for every pair of u,v $\in \mathcal{V}$}}

We can define a $\langle,\rangle$by
$$\forall u,v\in\mathcal{V},|u,v|=\frac{1}{2}(\mid\mid u+v\mid\mid
-\mid\mid u\mid\mid-\mid\mid v\mid\mid)$$
Then we prove that the $\langle,\rangle$ is an inner product. Real positivity:
$$\forall v\in\mathcal{V},\langle v,v\rangle=\frac{1}{2}(\mid\mid 2v\mid\mid
-\mid\mid v\mid\mid-\mid\mid v\mid\mid)=
\mid\mid v\mid\mid\geq 0$$
Definiteness:
\begin{equation*}
\langle v,v\rangle =0\Rightarrow 
\mid\mid v\mid\mid^2=\langle v,v\rangle=0	
\Rightarrow v = 0
\end{equation*}
\begin{equation*}
v=0\Rightarrow\langle v,v\rangle=\mid\mid v\mid\mid^2=0$$
\end{equation*}
Additivity in the first slot:
\begin{equation*}
	\begin{split}
	\forall u,v,w\in\mathcal{V},
	\langle v+u,w\rangle=\frac{1}{2}(\mid\mid u+v+2\frac{1}{2}w\mid\mid^2
	-\mid\mid u+v\mid\mid^2-\mid\mid w\mid\mid^2)\\=
	\frac{1}{2}(2\mid\mid u+\frac{1}{2}w\mid\mid^2+
	2\mid\mid v+\frac{1}{2}w\mid\mid^2-
	\mid\mid u-v\mid\mid-\mid\mid u+v\mid\mid-\mid\mid w\mid\mid)\\=
	\frac{1}{2}(2\mid\mid u+\frac{1}{2}w\mid\mid^2+2\mid\mid w\mid\mid^2
	+\mid\mid v+\frac{1}{2}w\mid\mid^2+2\mid\mid w\mid\mid-
	\mid\mid u+v\mid\mid^2-\mid\mid u-v\mid\mid^2-
	2\mid\mid w\mid\mid^2)\\=
	\frac{1}{2}(\mid\mid u+w\mid\mid^2+\mid\mid u\mid\mid^2
	-2\mid\mid u\mid\mid^2-2\mid\mid v\mid\mid^2
	-2\mid\mid w\mid\mid^2)\\=
	\frac{1}{2}(\mid\mid u+w\mid\mid^2-\mid\mid u\mid\mid^2-\mid\mid w\mid\mid^2)
	+\frac{1}{2}(\mid\mid v+w\mid\mid^2-\mid\mid v\mid\mid^2-\mid\mid w\mid\mid^2)=
	\langle u,w\rangle+\langle v,w\rangle
	\end{split}
\end{euqation*}
Conjugate symmetry:
\begin{equation*}
	\forall u,v\in\mathcal{V},\langle u,v\rangle=
	\frac{1}{2}(\mid\mid u+v\mid\mid^2+\mid\mid u\mid\mid^2+\mid\mid v\mid\mid^2)=
	\langle v,u\rangle=\overline{\langle v,u\rangle}
\end{equation*}
Homogeneity in the first slot:
$$\forall u,v\in\mathcal{V}$$
Firstly, we prove the situation when $a=-1$ 
\begin{equation*}
	\langle -u,v\rangle
	=\langle 0,v\rangle-\langle u,v\rangle
	=-\langle u,v\rangle.
\end{equation*}
Thus, we can let a>0,for the situation when $a<0$, we replace
u by -u, then it turns into situation when $a>0$.
Then we prove $a\in\mathbb{Z}$.
\begin{equation*}
	\forall a>0\in\mathbb{Z},\langle au,v\rangle
	=\sum{i=1}{a}\langle u,v\rangle
	=a\langle u,v\rangle
\end{euqation*} 
Then when $a\in\mathbb{Q}$.
\begin{equation*}
	\forall a=\frac{p}{q}\in\mathbb{Q},p,q\in\mathbb{Z},
	\frac{p}{q}\langle u,v\rangle
	=\frac{1}{q}\langle \frac{pq}{q}u,v\rangle
	=\frac{1}{q}q\langle \frac{p}{q}u,v\rangle
	=\langle \frac{p}{q}u,v\rangle
\end{equation*}
Due to the continuity of $\langle u,v\rangle$. We can span a to irrational numbers.
	\begin{gather*}
	\forall a\in\mathbb{R},\forall \epsilon>0,
	\exists \delta, \forall p\in\mathbb{Q},\mid p-a\mid<delta,\\
	\mid a\langle u,v\rangle-p\langle u,v\rangle\mid<\epsilon,
	\mid \langle au,v\rangle-\langle pu,v\rangle\mid<\epsilon\\
	\because p\langle u,v\rangle=\langle pu,v\rangle\\
	\therefore a\langle u,v\rangle=\langle au,v\rangle
	\end{gather*}
Therefore, we have a inner product$\langle,\rangle:\mathcal{V}\rightarrow
\mathbb{R}.$

\section*{VIII. \small{Tell a story about determinants from the viewpoint of problem-driven abstraction.}}
Inorder to deter whether the N-element linear equations $AX=0$
has a nonzero solution,
where $A={aij} \in{\mathbb{R}}^{n\times n},X\in{R}^n$,
Leibniz introduced the conception of determinant,
he proved that $determant=(-1)^{\tau t_{1}t_{2}...t_{n}}\Pi 
a_{1t_{1}}a_{2t_{2}}...a_{nt_{n}}$($\{t_{i}\}$is a 
arrange of 1 to n).
If $determinant = 0$, then the N-element linear equations has a nonzero solution .  
After the apperance of matrix, determinant is symbolized by
$detA$, which is called the determinant of matrix A.
Since we can view A as $(v_{1},v_{2},...,v_{n}),v_{i}\in{R}^{n}$,
its easy to prove that the definition of determinant
in our material is also meeted.

We can also view $A$ as a linear map which maps 
$\mathbb{P}^{n}$ to $\mathbb{P}^{n}$.
Then we can easily turn
N-element linear equations $AX=0$ to  
$A$ maps $X$ to $0$,
which develops the abstract define of determinant 
to the geometric meaning.
The determinant of a linearing map $A$
means the stretch factor of the linear map $A$.
Its value shows the volume of a unit n-dimensional cuboid
mapped from the formal coordinate by $A$.
So we can take $\mid .\mid$ to symbolize $det A$ as a volume of 
linear map A.
If $\mid A\mid=0$, it means A can map $\mathbb{R}^{n}$ 
to $\mathbb{R}^{n-1}$, thus the volume of a n-dimensional cuboid
becomes zero.
The sign of $\mid A\mid$ means the direction of map $A$.
If $\mid A\mid<0$, it maps a left-handed coordinate 
into a right-handed one, or maps a right-handed coordinate
into a left-handed one.
If $\mid A\mid>0$, it does not change such a property of
the coordinate.

Since $detA \in\mathbb{R}$
, we can define the linearing orderings
by the value of determinant.

If we view $detA$ as the stretch factor of a linearing map,
we can define the linearing orderings of determinant:X is
a unit n-dimensional cuboid, $\forall A,B\in\mathbb{R}^{n}$,
if the directed volume of $AX \geq BX$, then $detA \geq detB$.


\end{document}

%%% Local Variables: 
%%% mode: latex
%%% TeX-master: t
%%% End: 
