\documentclass[twoside,a4paper]{article}
\usepackage{geometry}
\geometry{margin=1.5cm, vmargin={0pt,1cm}}
\setlength{\topmargin}{-1cm}
\setlength{\paperheight}{29.7cm}
\setlength{\textheight}{25.3cm}

% useful packages.
\usepackage{amsfonts}
\usepackage{amsmath}
\usepackage{amssymb}
\usepackage{amsthm}
\usepackage{enumerate}
\usepackage{graphicx}
\usepackage{multicol}
\usepackage{fancyhdr}
\usepackage{layout}

% some common command
\newcommand{\dif}{\mathrm{d}}
\newcommand{\avg}[1]{\left\langle #1 \right\rangle}
\newcommand{\difFrac}[2]{\frac{\dif #1}{\dif #2}}
\newcommand{\pdfFrac}[2]{\frac{\partial #1}{\partial #2}}
\newcommand{\OFL}{\mathrm{OFL}}
\newcommand{\UFL}{\mathrm{UFL}}
\newcommand{\fl}{\mathrm{fl}}
\newcommand{\op}{\odot}
\newcommand{\Eabs}{E_{\mathrm{abs}}}
\newcommand{\Erel}{E_{\mathrm{rel}}}

\begin{document}

\pagestyle{fancy}
\fancyhead{}
\lhead{NAME Jiatu Yan}
\chead{Numerical Analysis homework \#1}
\rhead{Date 2020.3.20}


\section*{I. \small{Binary Method}}
I used binary method found the roots of the following 
four functions, the iterating times and the times of postcondition $\epsilon$
are nearly in linear relationship. The converging speed is 1, not so fast.

\subsection*{I-a \small{$x^{-1}-\tan x $ on $[0,\frac{\pi}{2}]$}} 

\begin{tabular}{|c|c|c|}
\hline
postcondition $\epsilon$ and $\delta$ & $\alpha$ & iterating times\\
\hline
$10^{-5}$ & 0.860335521730643  &     17\\
\hline
$10^{-6}$ & 0.860333274688473  &     21\\
\hline
$10^{-7}$ & 0.860333555568744  &     24\\
\hline
$10^{-8}$ & 0.860333590678778  &     27\\
\hline
$10^{-9}$ & 0.860333589215860  &     30\\
\hline
$10^{-10}$ & 0.860333589032996  &     33\\
\hline
$10^{-11}$ & 0.860333589021567  &     37\\
\hline
$10^{-12}$ & 0.860333589019424  &     41\\
\hline
$10^{-13}$ & 0.860333589019334  &     44\\
\hline
$10^{-14}$ & 0.860333589019379  &     45\\
\hline
\end{tabular}

\subsection*{I-b \small{$x^{-1}-2^{x}$ on $[0,1]$}} 

\begin{tabular}{|c|c|c|}
\hline
postcondition  $\epsilon$ and $\delta$ & $\alpha$  & iterating times\\
\hline
$10^{-5}$ & 0.641181945800781 & 17\\
\hline
$10^{-6}$ & 0.641185760498047 & 18\\
\hline
$10^{-7}$ & 0.641185760498047 & 18\\
\hline
$10^{-8}$ & 0.641185745596886 & 26\\
\hline
$10^{-9}$ & 0.641185744665563 & 30\\
\hline
$10^{-10}$ & 0.641185744490940 & 34\\
\hline
$10^{-11}$ & 0.641185744505492 & 36\\
\hline
$10^{-12}$ & 0.641185744504583 & 40\\
\hline
$10^{-13}$ & 0.641185744504980 & 44\\
\hline
$10^{-14}$ & 0.641185744504988 & 47\\
\hline
\end{tabular}

\subsection*{I-c \small{$2^{-x}+e^{x}+2\cos x-6$ on $[1,3]$}}
\begin{tabular}{|c|c|c|}
\hline
postcondition  $\epsilon$ and $\delta$ & $\alpha$  & iterating times\\
\hline
$10^{-5}$ & 1.829383850097656 & 18\\
\hline
$10^{-6}$ & 1.829382896423340 & 21\\
\hline
$10^{-7}$ & 1.829383611679077 & 23\\
\hline
$10^{-8}$ & 1.829383604228497 & 28\\
\hline
$10^{-9}$ & 1.829383601434529 & 31\\
\hline
$10^{-10}$ & 1.829383601958398 & 35\\
\hline
$10^{-11}$ & 1.829383601936570 & 38\\
\hline
$10^{-12}$ & 1.829383601933841 & 41\\
\hline
$10^{-13}$ & 1.829383601933841 & 41\\
\hline
$10^{-14}$ & 1.829383601933849 & 48\\
\hline
\end{tabular}

\subsection*{I-d \small{$\left( x^{3} + 4x^2+3x+5 \right)
/\left( 2x^{3}-9x^2+18x-2 \right) $ on $[0,4]$}}
\begin{tabular}{|c|c|c|}
\hline
postcondition  $\epsilon$ and $\delta$  & $\alpha$  & iterating times\\
\hline
$10^{-5}$ & 0.117881774902344 & 19\\
\hline
$10^{-6}$ & 0.117877006530762 & 22\\
\hline
$10^{-7}$ & 0.117876589298248 & 26\\
\hline
$10^{-8}$ & 0.117876566946507 & 29\\
\hline
$10^{-9}$ & 0.117876566015184 & 32\\
\hline
$10^{-10}$ & 0.117876566771884 & 36\\
\hline
$10^{-11}$ & 0.117876566793711 & 39\\
\hline
$10^{-12}$ & 0.117876566794621 & 42\\
\hline
$10^{-13}$ & 0.117876566795360 & 46\\
\hline
$10^{-14}$ & 0.117876566795310 & 49\\
\hline
\end{tabular}

\section*{II. \small{Newton's Method}}

I used Newton's Method found the roots near 4.5 and 7.7 
of the following function.
It can be found that the Newton's Method converges so fast that
the root can be found only in iterating times in unit digits.
\subsection*{II-a \small{$x = \tan x$,  $x_0 = 4.5$}}
\begin{tabular}{|c|c|c|}
\hline
postcondition  $\epsilon$  & $\alpha$  & iterating times\\
\hline
$10^{-5}$ & 4.493409655013248 & 2\\
\hline
$10^{-6}$ & 4.493409457909247 & 3\\
\hline
$10^{-7}$ & 4.493409457909247 & 3\\
\hline
$10^{-8}$ & 4.493409457909247 & 3\\
\hline
$10^{-9}$ & 4.493409457909247 & 3\\
\hline
$10^{-10}$ & 4.493409457909247 & 3\\
\hline
$10^{-11}$ & 4.493409457909247 & 3\\
\hline
$10^{-12}$ & 4.493409457909064 & 4\\
\hline
$10^{-13}$ & 4.493409457909064 & 4\\
\hline
$10^{-14}$ & 4.493409457909064 & 4\\
\hline
\end{tabular}

\subsection*{II-b \small{$x = \tan x $, $x_0 = 7.7$}}
There is a problem that when $\epsilon = 10^{-14}$, the iteration does not stop untill the iteration times outof the postcondition. I think the problem happens because the finit digits of cpp program, the $x_n$ is so close to the root that after  $f
\left( x_n \right)$ is rounded up in double 
, $f(x_{n}) = f(x_{n+1})$ and $ f(x_n) > 10^{-14}$, that may be 
why the iteration will not stop.

\begin{tabular}{|c|c|c|}
\hline
postcondition  $\epsilon$ and $\delta$ & $\alpha$  & iterating times\\
\hline
$10^{-5}$ & 7.725251836938464 & 4\\
\hline
$10^{-6}$ & 7.725251836938464 & 4\\
\hline
$10^{-7}$ & 7.725251836938464 & 4\\
\hline
$10^{-8}$ & 7.725251836938464 & 4\\
\hline
$10^{-9}$ & 7.725251836938464 & 4\\
\hline
$10^{-10}$ & 7.725251836938464 & 4\\
\hline
$10^{-11}$ & 7.725251836937707 & 5\\
\hline
$10^{-12}$ & 7.725251836937707 & 5\\
\hline
$10^{-13}$ & 7.725251836937707 & 5\\
\hline
$10^{-14}$ & 7.725251836937707 & 100\\
\hline
\end{tabular}

\section*{III. \small{Secant Method}}

I used Secant Method found the root of the following functions.
It can be found that the convergent speed of Secant Method
is also very fast. But it is slower than Newton's Method. 
\subsection*{III-a \small{$\sin \left( \frac{x}{2} \right)-1 $
with $x_0=0,x_1=\frac{\pi}{2}$}}

\begin{tabular}{|c|c|c|}
\hline
postcondition   $\epsilon$  and  $\delta$  & $\alpha$  & iterating times\\
\hline
$10^{-5}$ & 2.221441469079183 & 1\\
\hline
$10^{-6}$ & 2.579569691806805 & 2\\
\hline
$10^{-7}$ & 3.134268450825327 & 11\\
\hline
$10^{-8}$ & 3.138795060236532 & 13\\
\hline
$10^{-9}$ & 3.140932231503763 & 16\\
\hline
$10^{-10}$ & 3.141340394802018 & 18\\
\hline
$10^{-11}$ & 3.141533103369373 & 21\\
\hline
$10^{-12}$ & 3.141569907429061 & 23\\
\hline
$10^{-13}$ & 3.141583965313214 & 25\\
\hline
$10^{-14}$ & 3.141590602603975 & 28\\
\hline
\end{tabular}

\subsection*{III-b \small{$e^{x}-\tan x$ with $x_0=0,x_1=1.4$}}

\begin{tabular}{|c|c|c|}
\hline
postcondition   $\epsilon$  and  $\delta$  & $\alpha$  & iterating times\\
\hline
$10^{-5}$ & 1.159924363529063 & 1\\
\hline
$10^{-6}$ & 1.121423255162759 & 3\\
\hline
$10^{-7}$ & 1.306326943021557 & 13\\
\hline
$10^{-8}$ & 1.306326943021557 & 13\\
\hline
$10^{-9}$ & 1.306326943021557 & 13\\
\hline
$10^{-10}$ & 1.306326940423042 & 14\\
\hline
$10^{-11}$ & 1.306326940423042 & 14\\
\hline
$10^{-12}$ & 1.306326940423042 & 14\\
\hline
$10^{-13}$ & 1.306326940423042 & 14\\
\hline
$10^{-14}$ & 1.306326940423042 & 14\\
\hline
\end{tabular}

\subsection*{III-c \small{$x^{3}-12x^2+3x+1$ with $x_0=0,x_1=-0.5$}}
\begin{tabular}{|c|c|c|}
\hline
postcondition   $\epsilon$  and  $\delta$  & $\alpha$  & iterating times\\
\hline
$10^{-5}$ & -0.108108108108108 & 1\\
\hline
$10^{-6}$ & -0.182578753220966 & 3\\
\hline
$10^{-7}$ & -0.188685400608274 & 6\\
\hline
$10^{-8}$ & -0.188685400608274 & 6\\
\hline
$10^{-9}$ & -0.188685400608274 & 6\\
\hline
$10^{-10}$ & -0.188685403446523 & 7\\
\hline
$10^{-11}$ & -0.188685403446523 & 7\\
\hline
$10^{-12}$ & -0.188685403446523 & 7\\
\hline
$10^{-13}$ & -0.188685403446523 & 7\\
\hline
$10^{-14}$ & -0.188685403446523 & 7\\
\hline
\end{tabular}

\end{document}

%%% Local Variables: 
%%% mode: latex
%%% TeX-master: t
%%% End: 
