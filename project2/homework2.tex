\documentclass[twoside,a4paper]{article}
\usepackage{geometry}
\geometry{margin=1.5cm, vmargin={0pt,1cm}}
\setlength{\topmargin}{-1cm}
\setlength{\paperheight}{29.7cm}
\setlength{\textheight}{25.3cm}

% useful packages.
\usepackage{amsfonts}
\usepackage{amsmath}
\usepackage{amssymb}
\usepackage{amsthm}
\usepackage{enumerate}
\usepackage{graphicx}
\usepackage{multicol}
\usepackage{fancyhdr}
\usepackage{layout}

% some common command
\newcommand{\dif}{\mathrm{d}}
\newcommand{\avg}[1]{\left\langle #1 \right\rangle}
\newcommand{\difFrac}[2]{\frac{\dif #1}{\dif #2}}
\newcommand{\pdfFrac}[2]{\frac{\partial #1}{\partial #2}}
\newcommand{\OFL}{\mathrm{OFL}}
\newcommand{\UFL}{\mathrm{UFL}}
\newcommand{\fl}{\mathrm{fl}}
\newcommand{\op}{\odot}
\newcommand{\Eabs}{E_{\mathrm{abs}}}
\newcommand{\Erel}{E_{\mathrm{rel}}}

\begin{document}

\pagestyle{fancy}
\fancyhead{}
\lhead{NAME Jiatu Yan}
\chead{Numerical Analysis homework \#2}
\rhead{Date 2020.3.24}


\section*{I. \small{Consider the case $f\left( x \right)=\frac{1}{x}
,x_0=1,x_1=2$}}

\subsection*{I-a \small{Determine $\xi\left( x \right) $ explicitly.}} 

$f[1]=f\left( 1 \right)=1 $,$x[2]=f\left( 2 \right)=\frac{1}{2} $
, we have $f[1, 2]=\frac{\frac{1}{2}-1}{2-1}=-\frac{1}{2}$.
So $p_{1}\left( f;x \right)=1-\frac{1}{2}\left( x-1 \right)$.
Since $f''\left( x \right)=2 \frac{1}{x^{3}} $. Put it into the equation 
\[
	\f\left( x \right)-p_{1}\left( f;x \right) 
	=\frac{f''\left( \xi\left( x \right)  \right) }{2}
	\left( x-1 \right)\left( x-2 \right)  
.\] 
we have
\[
	\frac{1}{x}-\frac{3}{2}+\frac{x}{2}
	=\frac{1}{\left(  \xi\left( x \right) \right)^{3} \left( x-1 \right)\left(  x-2\right)  }
.\] 
Thus,
\[
	\xi\left( x \right)=\sqrt[3]{2x}     
.\] 


\subsection*{I-b \small{For $[x_1, x_2]$, find max$\xi\left( x \right)$ 
, min$\xi\left( x \right) $, and max$f''\left( \xi\left( x \right)  \right) $.}}

On $[1, 2]$, we have $\xi\left( x \right)=\sqrt[3]{2x}$, which is monotone increasing. So we have max$\xi\left( x \right)=\sqrt[3]{4}  $ 
, min$\xi\left(  x\right)=\sqrt[3]{2}  $.
$f''\left(  \xi\left( x \right) \right) = 4x$, which is also monotone increasing, we have max$f''\left( \xi\left( x \right)  \right) =8 $8 

\section*{II. \small{Find $p\in\mathbb{P}_{2n}^{+}$ such that  $p\left( x_{i} \right)=f_{i} $ for $i=0,1,\ldots,n$ where $f_{i}\ge0$ 
and $x_{i}$ are distinct points on $\mathbb{R}$}}

We take 
\[
	p
	=\sum_{i=0}^{n}\frac{f_{i}\prod_{k\neq i}\left(  x-x_{k}\right)^{2}}{\prod_{k\neq i}{\left( x_{i}-x_{k} \right)^{2} }} 
.\] 
It is obviously that $p\in\mathbb{P}_{2n}^{+}$ and $\forall i=0\ldots n, p\left(  x_{i}\right)=f_{i} $. So we find the p.

\section*{III. \small{Consider $f\left( x \right)=e^{x} $}}

\subsection*{III-a \small{Prove by induction that $\forall t\in R,
f[t,t+1,\ldots,t+n]=\frac{\left(  e-1\right)^{n} }{n!}e^{t}$.}}

When $n=1$, it is obvious that  $f[t]=f\left( t \right)=e^{t} $.

We assume that when $n=k$,  $\forall t\in\mathbb{R}$, $f[t,t+1,\ldots,t+n]=\frac{\left( e-1 \right)^{n} }{n!}e^{t}$.

Then when $n=k+1$, we have
\begin{equation*}
\begin{split}
	f[t,\ldots,t+n+1] &=\frac{f[t+1,\ldots,t+n+1]-f[t,\ldots,t+n]}{t+n+1-t}
		       \\ &= \frac{\frac{\left( e-1 \right)^{n} }{n!}\left( e^{t+1}-e^{t} \right) }{n+1}
			\\&=\frac{\left( e-1 \right)^{n+1} }{\left( n+1 \right)! }e^{t}
 \end{split}
 \end{equation*}
 
 So we proved that 
$f[t,t+1,\ldots,t+n]=\frac{\left(  e-1\right)^{n} }{n!}e^{t}$.

\subsection*{III-b \small{We know 
		$\xi\in\left( 0,n \right) $ s.t. $f[0,1,\ldots,n]=\frac{1}{n}f^{n}\left( \xi \right) $.
Determine $\xi$ from the above two equations. And is $\xi$ smaller or larger than  $\frac{n}{2}$?}}

Since $f^{\left( n \right) }\left( x \right)=e^{x} $,
\[
	f[0,1,\ldots,n]=\frac{\left( e-1 \right)^{n} }{n!}e^{0}=\frac{\left( e-1 \right)^{n} }{n!}=\frac{1}{n!}e^{\xi}
.\] 

So $\xi=n\ln\left( e-1 \right) $. Since $\ln\left( e-1 \right)>\frac{1}{2} $, we have $\xi>\frac{n}{2}$. 
So $\xi$ locates to the right of the midpoint $\frac{n}{2}$.

\section*{IV \small{Consider $f\left( 0 \right)=5,
f\left(  1\right)=3, f\left(  3\right)=5,f\left( 4 \right)=12  $.}}

\subsection*{IV-a \small{Use the Newton formula to obtain $p_{3}\left( f;x \right) $.}}

\begin{tabular}{c|cccc}
x\\
0&5\\
1&3 &-2\\
3&5 &1 &1\\
4&12&7 &2&1/4\\
\end{tabular}

So we have $f[0]=5,f[0,1]=-2,f[0,1,3]=1,f[0,1,3,4]=\frac{1}{4}$, thus
\begin{equation*}
\begin{split}
	p_{3}\left( f;x \right)&=f[0]+f[0,1]x+f[0,1,3]x\left( x-1 \right)+f[0,1,3,4]x\left( x-1 \right)\left( x-3 \right)\\
			       &=5-2x+x\left( x-1 \right)+\frac{1}{4}x\left( x-1 \right)\left( x-3 \right)\\
			       &=\frac{1}{4}x^{3}-\frac{9}{4}x+5
\end{split}
\end{equation*}

\subsection*{IV-b \small{Find an approximate value for the location $x_{min}$ of the minimum.}}

To find the minimum of f,
$$p_{3}'{f;x}=\frac{3}{4}x^{2}-\frac{9}{4}=0 \Rightarrow x=\pm \sqrt{3} $$
When $x=\sqrt{3} $,
$$p_{3}''\left( f;x \right)=\frac{3}{2}x=\frac{3\sqrt{3} }{2}>0 $$
So $x_{min}=\sqrt{3} $. 

\section*{V \small{Consider $f\left( x \right)=x^{7} $}}

\subsection*{V-a \small{Compute $f[0,1,1,1,2,2]$.}}

$f'\left(  x\right)=7x^{6},f''\left( x \right)=42x^{5}  $, so $f[1,1]=f'\left( 1 \right)=7
,f[1,1,1]=\frac{1}{2}f''\left( 1 \right)=21,f[2,2]=f'\left( 2 \right)=448   $, we have
\begin{tabular}{c|cccccc}
x\\
0 &0\\
1 &1  &1\\
1 &1  &7  &6\\
1 &1  &7  &21 &15\\
2 &128&127&120&99 &42\\
2 &128&448&321&201&102&31\\
\end{tabular}

So we have $f[0,1,1,1,2,2]=30$.

\subsection*{V-b \small{Determine $\xi$.}}

By the equation $f[0,1,1,1,2,2]=\frac{f^{\left( 5 \right)}\left( \xi \right)}{5!},\xi\in\left( 0,2 \right) $
, $f^{\left( 5 \right)}\left( x \right) =7\times6\times5\times4\times3x^{2} $, we have
\[
	30=f[0,1,1,1,2,2]=\frac{7\times6\times5\times4\times3\xi^{2}}{5!}\Rightarrow\xi=\sqrt{\frac{10}{7}}  
.\] 

\section*{VI \small{f is a function for which we have 
$f\left( 0 \right)=1,f\left( 1 \right)=2,f'\left(  1\right)=-1,f\left(  3\right)=f'\left( 3 \right)=0$.}}

\subsection*{VI-a \small{Estimate $f\left( 2 \right)$ using Hermite interpolation .}}

\begin{tabular}{c|ccccc}
x\\
0&1\\
1&2&1\\
1&2&-1&-2\\
3&0&-1& 0           &$\frac{2}{3}$\\
3&0&0 &$\frac{1}{2}$&$\frac{1}{4}$&$-\frac{5}{36}$ \\
\end{tabular}

So $f[0]=1,f[0,1]=1,f[0,1,1]=-2,f[0,1,1,3]=\frac{2}{3},f[0,1,1,3,3]=\frac{5}{36}$. Then we have
\[
	p\left( f;x \right)=1+x-2x\left( x-1 \right)+\frac{2}{3}x\left( x-1 \right)^2-\frac{5}{36}x\left( x-1 \right)^2\left( x-3 \right)     
.\] 
So we can estimate $f\left( 2 \right) $ by $p\left( f;x \right) $ that 
\[
	f\left( 2 \right)=p\left(  f;2\right)=1+2-2\times2+\frac{2}{3}\times2-\frac{5}{36}\times 2\times\left( -1 \right)=\frac{11}{18}   
.\] 
\subsection*{VI-b \small{Estimate the maximum possible error of the above answer.}}

Due to the equation $f\left( x \right)-p\left( f;x \right)=\frac{f^{\left( 5 \right) }\left( \xi \right) }{5!}x\left( x-1 \right)^2\left( x-3 \right)^2$, we have
\[
	e\left( 2 \right)=\frac{f^{\left( 5 \right) }\left( \xi \right) }{5!}\times 2\times\left( -1 \right)  ,\xi\in[0,3]
.\] 
Since $ \mid  f^{\left( 5 \right) }\left( x \right) \mid\le M$ on $[0,3]$, we have
\[
	 \mid e\left( 2 \right) \mid \le \frac{2M}{5!}  
.\]
So the maximum possible error of $f\left( 2 \right) $ is $\frac{M}{60}$.

\section*{VII \small{Define forward and backward different, prove 
$\bigtriangleup^{k}f\left( x \right)=k!h^{k}f[x_{0},\ldots,x_{k}] $,
$\bigtriangledown^{k}f\left( x \right)=k!h^{k}f[x_{0},\ldots,x_{-k}] $}}

\subsection*{VII-a \small{Prove the forward difference}}
When $k=0$, it is obvious that $f\left( x \right)=f[x_0]$.

We suppose that when $k=n$, the equation is right.

Then when  $k=n+1$, 
\begin{equation*}
\begin{split}
\bigtriangleup^{n+1}f\left( x \right)&=\bigtriangleup^{n}f\left( x+h \right)-\bigtriangleup^{n}f\left( x \right)\\
&=n!h^{n}\left( f[x_0+h,\ldots,x_{n}+h]-f[x_0,\ldots x_{n}] \right)\\
&=n!h^{n}\left( f[x_1,\ldots,x_{n+1}]-f[x_0,\ldots,x_{n}] \right)\\ 
&=n!h^{n}f[x_0,\ldots,x_{n+1}]\left( n+1 \right)h\\
&=\left( n+1 \right)!h^{n+1}f[x_0,x_1,\ldots,x_{n+1}] 
\end{split}
\end{equation*}

So we proved that $\forall k\in\mathbb{N}$, 
$\bigtriangleup^{k}f\left( x \right)=k!h^{k}f[x_{0},x_1,\ldots,x_{k}] $

\subsection*{VII-b \small{Prove the backward difference}}

When $k=0$, it is obvious that $f( x )=f[x_0]$ .

We suppose that when $k=n$, the equation is right.

Then when $k=n+1$,
\begin{equation*}
\begin{split}
\bigtriangledown^{n+1}f\left( x \right) &=\bigtriangledown^{n}f\left( x \right) -\bigtriangledown^{n}f\left( x-h \right) \\
&=n!h^{n}\left( f[x_0,\ldots,x_{-n}]-f[x_0-h,\ldots, x_{-n}-h]\right)\\
&=n!h^{n}\left( f[x_{-n},x_{-\left( n-1 \right) }\ldots,x_{0}]-f[x_{-n-1},\ldots,x_{-1}] \right)\\ 
&=n!h^{n}f[x_0,x_{-1},\ldots,x_{-\left( n+1 \right)}]\left( n+1 \right)h\\
&=( n+1 )!h^{n+1}f[x_0,x_{-1},\ldots,x_{-\left( n+1 \right) }] 
\end{split}
\end{equation*}

So we proved that $\forall k\in\mathbb{N}$,
$\bigtriangledown^{k}f\left( x \right) =k!h^{k}f[x_{0},x_{-1},\ldots,x_{-k}]$.  

\section*{VIII \small{Assume $f$ is differentiable at  $x_{0}$.
Prove $\frac{\partial}{\partial x_0}f[x_0,\ldots,x_n]=f[x_0,x_0,\ldots,x_n]$. And what about the partial derivative with respect to other variables?}}

\subsection*{VIII-a \small{Prove $\frac{\partial}{\partial x_0}f[x_0,\ldots,x_n]=f[x_0,x_0,\ldots,x_n]$}}

When $n=0$, it is obvious that  $\frac{\partial f[x_0]}{\partial x_0}=f'\left( x_0 \right)=f[x_0,x_0] $

We suppose that when $n=k$, $\frac{\partial}{\partial x_0}f[x_0,\ldots,x_{k}]=f[x_0,x_0,\ldots,x_k]$

Then when $n=k+1$, we have
\begin{equation*}
\begin{split}
	\frac{\partial}{\partial x_0}f[x_0,\ldots,x_{k+1}]
	&=\frac{\partial}{\partial x_0}\left( f[x_1,\ldots,x_{k+1}]-f[x_0,\ldots,x_{k}] \right)\frac{1}{x_{k+1}-x_0}\\
	&=\frac{f[x_1,\ldots,x_{k+1}]}{\left( x_{k+1}-x_0 \right)^2 }
	-\frac{f[x_0,x_0,\ldots,x_{k}]\left( x_{k+1}-x_0 \right)+f[x_0,\ldots,x_{k}] }{\left( x_{k+1}-x_0 \right)^2 }\\
	&=\frac{f[x_0,x_1,\ldots,x_{k+1}]}{x_{k+1}-x_0}-\frac{f[x_0,x_0,x_1,\ldots,x_{k}]}{x_{k+1}-x_0}\\
	&=f[x_0,x_0,x_1,\ldots,x_{k+1}]
\end{split}
\end{equation*}

So we have proved that $\forall n\in\mathbb{N}$, $\frac{\partial}{\partial x_0}f[x_0,\ldots,x_n]=f[x_0,x_0,\ldots,x_n]$.

\subsection*{VIII=b \small{What about the partial derivative with respect to one of the other variables?}}

Since changing the order of $x_0,x_1,\ldots,x_{n}$, will not change the value of $f[x_0,\ldots,x_{n}]$.
$\forall i\in{0,1,\ldots,n}$. Now we suppose that $f$ is differentiable at  $x_{i}$ 
and $\{x_{k_{n}}\}$ are a repermutation of $x_0,x_1,\ldots,x_{n}$ in which $x_{k_{0}}=x_{i}$.
So we have
\begin{equation*}
\begin{split}
	\frac{\partial}{\partial x_{i}}f[x_0,x_1,\ldots,x_{n}]&=\frac{\partial}{\partial x_{k_{0}}}f[x_{k_{0}},x_{k_{1}},\ldots,x_{n}]\\
							      &=f[x_{k_{0}},x_{k_{0}},x_{k_{1}},\ldots,x_{k_{n}}]
\end{split}
\end{equation*}

To have a more general equation we can resort $x_{k_{0}}, x_{k_{0}},x_{k_1},\ldots,x_{k_{n}}$, then we will get the equation

$$\forall i\in\{0,1,\ldots,n\},\frac{\partial}{\partial x_{i}}f[x_0,x_1,\ldots,x_{n}]=f[x_{0},\ldots,x_{i},x_{i},\ldots,x_{n}]$$

\end{document}

%%% Local Variables: 
%%% mode: latex
%%% TeX-master: t
%%% End: 
