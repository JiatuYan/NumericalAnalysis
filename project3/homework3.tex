\documentclass[twoside,a4paper]{article}
\usepackage{geometry}
\geometry{margin=1.5cm, vmargin={0pt,1cm}}
\setlength{\topmargin}{-1cm}
\setlength{\paperheight}{29.7cm}
\setlength{\textheight}{25.3cm}

% useful packages.
\usepackage{amsfonts}
\usepackage{amsmath}
\usepackage{amssymb}
\usepackage{amsthm}
\usepackage{enumerate}
\usepackage{graphicx}
\usepackage{multicol}
\usepackage{fancyhdr}
\usepackage{layout}

% some common command
\newcommand{\dif}{\mathrm{d}}
\newcommand{\avg}[1]{\left\langle #1 \right\rangle}
\newcommand{\difFrac}[2]{\frac{\dif #1}{\dif #2}}
\newcommand{\pdfFrac}[2]{\frac{\partial #1}{\partial #2}}
\newcommand{\OFL}{\mathrm{OFL}}
\newcommand{\UFL}{\mathrm{UFL}}
\newcommand{\fl}{\mathrm{fl}}
\newcommand{\op}{\odot}
\newcommand{\Eabs}{E_{\mathrm{abs}}}
\newcommand{\Erel}{E_{\mathrm{rel}}}

\begin{document}

\pagestyle{fancy}
\fancyhead{}
\lhead{NAME Jiatu Yan}
\chead{Numerical Analysis homework \#3}
\rhead{Date 2020.3.29}


\section*{I. \small{For $n\in\mathbb{N^{+}}$, determin $\min\max \limits_{x\in[a,b]} \mid a_0x^{n}+a_1x^{n-1}+\ldots+a_n \mid $.}}

Since $\forall \{a_n\}$, 
$$\max \limits_{x\in[-1,1]} \mid x^{n}+a_1x^{n-1}+\ldots+a_n \mid\ge \frac{1}{2^{n-1}}.$$
And the equality can be reached if 
$x^{n}+a_1x^{n-1}+\ldots+a_n=\frac{1}{2^{n-1}}T_n\left( x \right) .$
Thus we can have a conversion of the fomular $a_0x^{n}+a_1x^{n-1}+\ldots+a_n$ that 
\[
	\max\limits_{x\in[a,b]} \mid a_0x^{n}+a_1x^{n-1}+\ldots+a_n \mid 
		= \max\limits_{x\in[-1,1]} \mid a_0\left( \frac{b-a}{2}x+\frac{b+a}{2} \right)^{n}+\ldots+a_n \mid 
		=\max\limits_{x\in[-1,1]} \mid a_0'x^{n}+a'_1x^{n-1}+\ldots+a'_n \mid 
.\] 
The $\{a'_n\}$ can be randomly taken because $\{a_n\}$ is random.
Because $a'_0$ is independent with $\{a'_n\}$, we suppose $a''_i=\frac{a'_i}{a'_0}$, we have
\begin{equation*}
	\begin{split}
	 \min\max\limits_{x\in[a,b]} \mid a_0x^{n}+a_1x^{n-1}+\ldots+a_n \mid 
	 &=\min|a_0|\max\limits_{x\in[-1,1]} \mid x^{n}+a''_1x^{n-1}+\ldots+a''_n \mid 
       \\&\ge \min|a_0|\frac{1}{2^{n-1}} .
\end{split}
\end{equation*}
Since the equality can be reached and $a_0$ is randomly taken from $\mathbb{R}\backslash\{0\}$, we have\
 \[
\min \mid a_0 \mid \frac{1}{2^{n-1}}=0
.\] 
Thus,
\[
\min\max \limits_{x\in[a,b]} \mid a_0x^{n}+a_1x^{n-1}+\ldots+a_n \mid =0
.\] 

\section*{II. \small{Prove $\forall p\in\mathbb{P}_n^{\alpha}, 
\lVert \hat{p}_n\left( x \right) \rVert_\infty\le \lVert p\rVert_\infty$.}}

Suppose that
$$\exists p\in\widetilde{\mathbb{P}}_n\quad s.t.\quad \max\limits_{x\in[-1.1]} \mid p\left( x \right) \mid <\lVert \hat{p}_n\left( x \right) \rVert.  $$
$T_n\left( x \right) $ assumes its extreme $n+1$ times at the  points $x'_k=\cos \frac{k}{n}\pi$ $k=0,1,\ldots,n$.
Consider the polynomial $Q\left( x \right)=\frac{T_n\left( x \right) }{T_n\left( \alpha \right) }-p\left( x \right) $. We have
\[
	Q\left( x'_{k} \right)=\frac{\left( -1 \right)^{k} }{T_n\left( \alpha \right) }-p\left( x'_k \right)  \qquad  k=0,1\ldots,n  
.\]
Since 
$\mid p\left( x'_k \right) \mid <\lVert \hat{p}_n\left( x \right) \rVert =\frac{1}{T_n\left( \alpha \right) } $.
$Q\left( x \right) $ has alternating signs at these $n+1$ points, which means  $Q\left( x \right) $ must have $n$ zeros in $[-1,1]$.
But we know  $Q\left( \alpha \right)=0, \alpha>1$ and $Q\left( x \right) $ is a polynomial with degree $n$. 
So  $Q\left( x \right) $ has at most $n-1$ zeros within  $[-1,1]$.
Therefore,  $Q\left( x \right)\equiv 0$ and $p\left( x \right)=\frac{T_n\left( x \right) }{T_n\left( \alpha \right) } $
, which means $\lVert \hat{p}_n\left( x \right) \rVert_\infty=\lVert p\rVert_\infty$. 
It is a contradition to our assumption.
Therefore, we have
\[
\forall p\in\mathbb{P}_n^{\alpha}, \lVert \hat{p}_n\left( x \right) \rVert_\infty\le \lVert p\rVert_\infty
.\] 
\end{document}

%%% Local Variables: 
%%% mode: latex
%%% TeX-master: t
%%% End: 
